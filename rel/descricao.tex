\newpage
\section{Descrição}
	Processamento de imagem é qualquer forma de processamento de dados no qual a entrada e saída são imagens tais como fotografias ou quadros de vídeo. Ao contrário do tratamento de imagens, que se preocupa somente na manipulação de figuras para sua representação final, o processamento de imagens é um estágio para novos processamentos de dados tais como aprendizagem de máquina ou reconhecimento de padrões. A maioria das técnicas envolve o tratamento da imagem como um sinal bidimensional, no qual são aplicados padrões de processamento de sinal.

\subsection{Métodos de processamento}
	Algumas décadas atrás o processamento de imagem era feito majoritariamente de forma analógica, através de dispositivos ópticos. Apesar disso, devido ao grande aumento de velocidades dos computadores, tais técnicas foram gradualmente substituídas por métodos digitais.
	
	O processamento digital de imagem é geralmente mais versátil, confiável e preciso, além de ser mais fácil de implementar que seus duais analógicos. Hardware especializado ainda é usado para o processamento digital de imagem, contando com arquiteturas de computador paralelas para tal, em sua maioria no processamento de vídeos. O processamento de imagens é, em sua maioria, feito por computadores pessoais.

\subsection{Técnicas mais usadas}
	A maioria dos conceitos de processamento de sinais que se aplicam a sinais unidimensionais também podem ser estendidos para o processamento bidimensional de imagens. A transformada de Fourier é bastante usada nas operações de imagem envolvendo uma grande área de correlação.
	
	\begin{itemize}
		\item Resolução de imagem;
		\item Limite dinâmico;
		\item Largura de banda;
		\item Filtro: Permite a redução de ruídos da imagem para que mais padrões possam ser encontrados;
		\item Operador diferencial;
		\item Histograma: Consiste na frequência de um tom específico (seja escala de cinza ou colorido) em uma imagem. Permite a obtenção de informações como o brilho e o contraste da imagem e sua distribuição;
		\item Detecção de borda;
		\item Redução de ruído.
	\end{itemize}

\subsection{Problemas típicos}
	Além de imagens bidimensionais estáticas, o campo também abrange o processamento de sinais variados pelo tempo tais como vídeos ou a saída de um equipamento de tomografia. Tais técnicas são especificadas somente para imagens binárias ou em escala de cinza.
	\begin{itemize}
		\item Transformações geométricas tais como escala, rotação e inclinação;
		\item Correção de cor como ajustes de brilho e contraste, limiarização ou conversão de espaço de cor;
		\item Combinação de imagens por média, diferença ou composição;
		\item Interpolação e recuperação de imagem de um formato bruto tal como o filtro bayesiano;
		\item Segmentação de uma imagem em regiões;
		\item Edição de imagem e acabamento (retoque) digital;
	\end{itemize}